\documentclass[10pt, letterpaper]{article}

\usepackage{hyperref}
\usepackage{geometry}
    \geometry{
        top=0.8in,
        bottom=0.8in,
        left=0.9in,
        right=0.9in
    }

% Comment the following lines to use the default Computer Modern font
% instead of the Palatino font provided by the mathpazo package.
% Remove the 'osf' bit if you don't like the old style figures.
\usepackage[T1]{fontenc}
\usepackage[sc,osf]{mathpazo}

% Set your name here
\def\name{Xiaodong Hu}

% Replace this with a link to your CV if you like, or set it empty
% (as in \def\footerlink{}) to remove the link in the footer:
\def\footerlink{}


% Customize page headers
\pagestyle{myheadings}
\markright{\name}
\thispagestyle{empty}

% Custom section fonts
\usepackage{sectsty}
\sectionfont{\rmfamily\mdseries\Large}
\subsectionfont{\rmfamily\mdseries\itshape\large}

% Other possible font commands include:
% \ttfamily for teletype,
% \sffamily for sans serif,
% \bfseries for bold,
% \scshape for small caps,
% \normalsize, \large, \Large, \LARGE sizes.

% Don't indent paragraphs.
\setlength\parindent{0em}

% Make lists without bullets
% \renewenvironment{itemize}{
%   \begin{list}{}{
%     \setlength{\leftmargin}{1.5em}
%   }
% }{
%   \end{list}
% }


\pagestyle{empty} % no title
\begin{document}
\vspace{-2in}
% place name at left
{\huge\bf\name}
\hfill\\

\begin{minipage}{0.5\linewidth}
    \begin{tabular}{ll}
         & Materials Science \& Engineering Department \\
         & 302 Roberts Hall, University of Washington  \\
         & Seattle, WA, 98195-2120
    \end{tabular}
\end{minipage}
\begin{minipage}{0.5\linewidth}
    \begin{tabular}{ll}
        Phone:    & (1) 857-272-7073                                                          \\
        Email:    & \href{mailto:hxd.phys@gmail.com}{\tt hxd.phys@gmail.com}                  \\
        Homepage: & \href{https://xiaodong-hu.github.io/}{\tt https://xiaodong-hu.github.io/} \\
    \end{tabular}
\end{minipage}

\makebox[\linewidth]{\rule{\linewidth}{1.0pt}}

\vspace{-0.5em}
\section*{Education}
\begin{minipage}{0.55\linewidth}
    \begin{tabular}{ll}
         & {\bf University of Washington} \\
         & 2024-present
    \end{tabular}
\end{minipage}
\begin{minipage}{0.45\linewidth}
    \begin{tabular}{ll}
         & Postdoc Scholar \\
    \end{tabular}
\end{minipage}
\\[1em]
\begin{minipage}{0.55\linewidth}
    \begin{tabular}{ll}
         & {\bf Boston College} \\
         & 2018-2024
    \end{tabular}
\end{minipage}
\begin{minipage}{0.45\linewidth}
    \begin{tabular}{ll}
         & Ph.D in Condensed Matter Theory \\
         & Advisor: \emph{Ying Ran}
    \end{tabular}
\end{minipage}
\\[1em]
\begin{minipage}{0.55\linewidth}
    \begin{tabular}{ll}
         & {\bf University of Science and Technology of China} \\
         & 2014-2018
    \end{tabular}
\end{minipage}
\begin{minipage}{0.45\linewidth}
    \begin{tabular}{ll}
         & B.S. in Theoretical Physics \\
         & { }
    \end{tabular}
\end{minipage}



% \section*{Research Interests}
% My research is primarily focused on exploring the intricate emergent phenomena in strongly-correlated systems, with a particular emphasis on the interplay of symmetry and topology, such as fractional Chern insulators (lattice analogue of fractional Quantum Hall effect), Kitaev materials, and high-$T_c$ superconductors. Both analytic methods and numerical simulations are used in my research.

\section*{Publications/Preprints}
\begin{itemize}
    \item \underline{X-D. Hu}, D. Xiao, and Y. Ran, \href{https://journals.aps.org/prb/abstract/10.1103/PhysRevB.109.245125}{\emph{Hyperdeterminants and Composite Fermion States in Fractional Chern Insulators}},  Phys. Rev. B \textbf{109}, 245125 (2024, Editor's Suggestion)
    \item \underline{X-D. Hu}, J-H. Han, and Y. Ran, \href{https://journals.aps.org/prb/abstract/10.1103/PhysRevB.108.L041106}{\emph{Supercurrent-induced anomalous thermal Hall effect as a new probe to superconducting gap anisotropy}}, Phys. Rev. B \textbf{108}, L041106 (2023)
    \item \underline{X-D. Hu}, and Y. Ran, \href{https://journals.aps.org/prb/abstract/10.1103/PhysRevB.106.125136}{\emph{Engineering chiral topological superconductivity in twisted Ising superconductors}}, Phys. Rev. B \textbf{106}, 125136 (2022)
    \item F. Bahrami, \underline{X-D. Hu}, Y. Du, O. I. Lebedev, C. Wang, H. Luetkens, G. Fabbris, M. J. Graf, D. Haskel, Y. Ran, and F. Tafti, \href{https://www.science.org/doi/full/10.1126/sciadv.abl5671}{\emph{First demonstration of tuning between the Kitaev and Ising limits in a honeycomb lattice}}, Sci. Adv. \textbf{8}, eabl5671 (2022)
\end{itemize}


\section*{Presentations}
\begin{itemize}
    \item \emph{Projective Construction of Fractional Chern Insulators}, Talk, APS March Meeting, 2024
    \item \emph{Projective Construction of Fractional Chern Insulators: Hyperdeterminants}, UW CMT Journal Club Talk, 2024
    \item \emph{Engineering Chiral Topological Superconductivity in Twisted Ising Superconductors}, Talk, APS March Meeting, 2023
    \item \emph{Supercurrent-induced anomalous thermal Hall effect as a new probe to superconducting gap anisotropy}, Talk, online APS March Meeting, 2023
\end{itemize}

\section*{Techniques}
\begin{tabular}{ll}
    Programming Languages & Julia, Python, Mathematica, Rust, C++. \\
    DFT Tools             & Quantum Espresso, ELK                  \\
\end{tabular}

% \section*{Developed Packages}
% \begin{itemize}
%     \item \texttt{FCI.jl}, includes \texttt{LLL.jl}, \texttt{MODEL.jl}, \texttt{FCB\_ED.jl}, \texttt{CF.jl}, \texttt{CF\_MF.jl}, \texttt{CF\_TDHF.jl}, \texttt{CF\_PROJWFC.jl}, etc. for our big project of the projective construction of fractional Chern insulators.
%     \item \texttt{TightBinding.jl} and \texttt{TightBindMeanField.jl} for tight-binding model construction, quantum transport study, and interface to \texttt{wannier90}.
% \end{itemize}

% \vspace{0.5em}
% I have developped julia packages \texttt{TightBinding.jl} for extracting tightbinding models from Quantum Espresso, and \texttt{TightBindingMeanField.jl} for mean-field calculation for cluster of crystals 


\section*{Teaching Experiences}
During my PhD time at Boston College, I have served as a Teaching Assistant for several graduate courses, including Classical Mechanics, Electrodynamics, Quantum Mechanics I/II, Statistical Mechanics I/II, Solid State Physics I, and Particle Physics. In the summer term of 2023, I also served as an instructor for an undergraduate course, Introduction to Physics I/II.



% Footer
\begin{center}
    \begin{footnotesize}
        Last updated: \today
    \end{footnotesize}
\end{center}

\end{document}

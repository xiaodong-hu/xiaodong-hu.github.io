%\usepackage[space=false]{ctex}
\usepackage{fontspec,xunicode,xltxtra}
\usepackage{hyperref}	
\usepackage{booktabs}	
\usepackage{mathrsfs,amssymb,amsfonts,amsmath,bm}	%Math packages
\usepackage{dsfont}     %double line number
\usepackage{color}
\usepackage{xcolor}
\usepackage{graphicx,psfrag}
\usepackage{epsfig}
\usepackage{array, booktabs}
\usepackage{graphicx}
\usepackage{caption}
\usepackage{slashed}	%Dirac Operator
\usepackage{tikz}
	\usetikzlibrary{calc}
	\usetikzlibrary{decorations.markings}
	\usetikzlibrary{arrows}
\usepackage{extarrows}


\def\Re{\mathop{\mathcal{R}e}}
\def\Im{\mathop{\mathcal{I}m}}
\def\arrow{\tikz[scale=0.1,baseline=.1ex]{
	\draw[fill=black,rotate=-90] (-0.7,0)--(0,2)--(0.7,0);}
	}

\def\cross{\tikz[scale=0.1,baseline=.1ex]{
	\draw[thick,rotate=45] (-1,0)--(1,0);
	\draw[thick,rotate=45] (0,-1)--(0,1);}
	}

\newcommand*\dd{\mathop{}\!\mathrm{d}}
\newcommand*\ddd[1]{\mathop{}\!\mathrm{d^#1}}
\def\Re{\mathop{\mathcal{R}e}}					%Real part
\def\Im{\mathop{\mathcal{I}m}}					%imaginary part
\def\imp{\text{imp}}

\mode <presentation>
%\usetheme{Madrid}%% default Warsaw [secheader]
%\usetheme{Warsaw}%% tested by hxd
\usetheme{CambridgeUS}
\usecolortheme{beaver} %% changed by hxd, red style
\beamersetaveragebackground{black!10} % gray scale background

\setbeamercovered{transparent}
%\setbeamertemplate{navigation symbols}{}

\usefonttheme{professionalfonts}
\useinnertheme{circles}%{rectangles}
\setbeamertemplate{itemize item}{$\circledast$}% design the item bullet %\circledast  %\checkmark


\DeclareCaptionFont{blue}{\color{orange}}%LightSteelBlue3

\newcommand{\foo}{\color{blue}\makebox[0pt]{\textbullet}\hskip-0.5pt\vrule width 1pt\hspace{\labelsep}}%LightSteelBlue3

\hypersetup{pdfpagemode=FullScreen} % makes your presentation go automatically to full screen
%\setcounter{tocdepth}{1} % depth of table of contents

\defbeamertemplate{title page}{English}{ % English version of title page, use by \setbeamertemplate{title page}[Chinese/English]
	%\vbox{}
	\vfill
	\begin{center}
		\includegraphics[height=2.0cm]{BC-Eagle.svg}
		\vskip1em\par%
		\begin{beamercolorbox}[sep=8pt,center,shadow=true,rounded=true]{title}
			\usebeamerfont{title}\inserttitle\par%
			\ifx\insertsubtitle\@empty%
			\else%
				\vskip0.25em%
				{\usebeamerfont{subtitle}\usebeamercolor[fg]{subtitle}\insertsubtitle\par}%
			\fi%
		\end{beamercolorbox}%
		%\vskip1em\par
		%\begin{beamercolorbox}[sep=8pt,center]{author}
		%	\usebeamerfont{author}%\insertauthor
		%	{
		%	    \begin{tabular}{cc}
		%	    Respondant: &\insertauthor\\ %答辩人:
		%	    Supervisor: &\advisor %导\quad 师:
		%	    \end{tabular}
		%	}
		\end{beamercolorbox}
		\begin{beamercolorbox}[sep=6pt,center]{institute} %sep means separationb between beamercolorboxes
			\usebeamerfont{institute}\insertinstitute
		\end{beamercolorbox}
		\begin{beamercolorbox}[sep=6pt,center]{date}
			\usebeamerfont{date}\insertdate
		\end{beamercolorbox}\vskip0.5em
		%{\usebeamercolor[fg]{titlegraphic}\inserttitlegraphic\par}
	\end{center}
	\vfill
}

%\setbeamertemplate{section in toc}[ball,hideothersubsections]
%\setbeamertemplate{subsection in toc}[subsections numbered]


\setbeamertemplate{footline}{% change the ratio of each colorbox, cf. https://tex.stackexchange.com/questions/315580/how-to-adjust-width-of-footnotes-in-beamer-cambridgeus-theme/315587
	\leavevmode%
	\hbox{%
		\begin{beamercolorbox}[wd=.25\paperwidth,ht=2.25ex,dp=1ex,center]{author in head/foot}%
			\usebeamerfont{author in head/foot}\insertshortauthor\expandafter~~(\insertshortinstitute)
		\end{beamercolorbox}%
		\begin{beamercolorbox}[wd=.45\paperwidth,ht=2.25ex,dp=1ex,center]{title in head/foot}%
			\usebeamerfont{title in head/foot}\insertshorttitle
		\end{beamercolorbox}%
		\begin{beamercolorbox}[wd=.3\paperwidth,ht=2.25ex,dp=1ex,right]{date in head/foot}%
			\usebeamerfont{date in head/foot}\insertshortdate{}\hspace*{2em}
			\insertframenumber{} / \inserttotalframenumber\hspace*{2ex} 
		\end{beamercolorbox}
	}%
	\vskip0pt%
}

%%%%%%%%% set environments like block's color %%%%%%%%%%%
	\setbeamercolor{block title}{use=structure,fg=structure.fg,bg=structure.fg!20!bg}
	\setbeamercolor{block body}{parent=normal text,use=block title,bg=block title.bg!50!bg}

	\setbeamercolor{block title example}{use=example text,fg=example text.fg,bg=example text.fg!20!bg}
	\setbeamercolor{block body example}{parent=normal text,use=block title example,bg=block title example.bg!50!bg}

	\newenvironment<>{redblock}[1]{%
		\setbeamercolor{block title}{fg=white,bg=red!70!pink}%
		\begin{block}#2{#1}}{\end{block}
	}
	\newenvironment<>{greenblock}[1]{%
		\setbeamercolor{block title}{fg=white,bg=green!70!blue}%
		\begin{block}#2{#1}}{\end{block}
	}

	\addtobeamertemplate{proof begin}{%
		\setbeamercolor{block title}{fg=black,bg=red!50!white}
		\setbeamercolor{block body}{fg=red, bg=red!30!white}
	}{}

	\BeforeBeginEnvironment{theorem}{
		\setbeamercolor{block title}{fg=black,bg=orange!50!white}
		\setbeamercolor{block body}{fg=orange, bg=orange!30!white}
	}
	\AfterEndEnvironment{theorem}{
		\setbeamercolor{block title}{use=structure,fg=structure.fg,bg=structure.fg!20!bg}
		\setbeamercolor{block body}{parent=normal text,use=block title,bg=block title.bg!50!bg, fg=black}
	}